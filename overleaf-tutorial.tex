% \documentclass[11pt]{letter}

% \begin{document}

% \end{document}

% //////////////////////

% \documentclass [12pt, letterpaper] {article}
% \usepackage[utf8]{inputenc}

% ////////////
% \title{First document}
% \author{Hubert Farnsworth}
% \date{\today}
% \documentclass[12pt, letterpaper]{article}
% \usepackage[utf8]{inputenc}

\documentclass[12pt, letterpaper, twoside]{article}
\usepackage[utf8]{inputenc}

\title{First document}
\author{Prahalad Belavadi \thanks{My mum, Prajwal and the Resumepuppy team}}
\date{\today}

\begin{document}

\maketitle
We have now added a title, author and date to our first \LaTeX{} document!

Mostly just copying this from the overleaf site 

First document. This is a simple example, with no 
extra parameters or packages included.

here's  \textbf{a bold sentence}

some \textit{italics too} 

this is how to \underline{underline}

% to emphasize
Some of the greatest \emph{discoveries} 
in science 
were made by accident.

\textit{Some of the greatest \emph{discoveries} 
in science 
were made by accident.}

\textbf{Some of the greatest \emph{discoveries} 
in science 
were made by accident.}


if only I coulld figure out how to type on the next line without massive line skips

% Need to figure out how to upload images
% \documentclass{article}
% \usepackage{graphicx}
% \graphicspath{ {images/} }

% \begin{document}
% The universe is immense and it seems to be homogeneous, 
% in a large scale, everywhere we look at.

% \includegraphics{universe}

% There's a picture of a galaxy above
% \end{document}


\begin{itemize}
    \item The individual entries are indicated with a black dot, a so-called bullet.
    \item The text in the entries may be of any length.
  \end{itemize}


  \begin{enumerate}
    \item This is the first entry in our list
    \item The list numbers increase with each entry we add
  \end{enumerate}


  In physics, the mass-energy equivalence is stated 
  by the equation $E=mc^2$, discovered in 1905 by Albert Einstein.


  The mass-energy equivalence is described by the famous equation
\[ E=mc^2 \]
discovered in 1905 by Albert Einstein. 
In natural units ($c = 1$), the formula expresses the identity
\begin{equation}
E=mc^2
\end{equation}


Subscripts in math mode are written as $a_b$ and superscripts are written as $a^b$. These can be combined an nested to write expressions such as

\[ T^{i_1 i_2 \dots i_p}_{j_1 j_2 \dots j_q} = T(x^{i_1},\dots,x^{i_p},e_{j_1},\dots,e_{j_q}) \]
 
We write integrals using $\int$ and fractions using $\frac{a}{b}$. Limits are placed on integrals using superscripts and subscripts:

\[ \int_0^1 \frac{dx}{e^x} =  \frac{e-1}{e} \]

Lower case Greek letters are written as $\omega$ $\delta$ etc. while upper case Greek letters are written as $\Omega$ $\Delta$.

Mathematical operators are prefixed with a backslash as $\sin(\beta)$, $\cos(\alpha)$, $\log(x)$ etc.
  
\end{document}

